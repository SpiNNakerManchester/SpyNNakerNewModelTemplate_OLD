\documentclass[12pt]{article}

\title{IF\_curr\_comb\_exp: combined exponential synaptic response}
\author{Oliver Rhodes}
\date{\today}


\begin{document}
\maketitle

\section{Synapse Dynamics}
Upon the presynaptic neuron firing, a synapse response is triggered, causing some input to be delivered to the post-synaptic neuron. This document covers the implementation of a current response shaped via two exponential functions. First the target shape is introduced, followed by details of an implementation on the SpiNNaker platform.

\begin{eqnarray}
i(t) = f(e^{-\frac{t}{\tau_a}} - e^{-\frac{-t}{\tau_b}}) \\
t_{rise} = ln\frac{B\tau_x}{A\tau_x}(\frac{\tau_x\tau_x}{\tau_x - \tau_x} \\
f = \frac{1}{e^{-\frac{t_{rise}}{\tau_a}} - e^{-\frac{-t_{rise}}{\tau_b}}} \\
\label{eqn:combined_exponential_response}
\end{eqnarray}

\section{SpiNNaker Implementation}
A generic combined exponential synapse is implemented to facilitate exploration of further synapse shapes defined through combinations of exopnential functions. The current $i(t)$ is therefore defined as in Eqn.\ref{eq:generic_form}. 
\begin{equation}
i(t) = Ae^{\frac{-t}{\tau_a} + Be^{\frac{-t}{\tau_b}
\frac{di(t)}{dt} = \frac{-A}{\tau_a}e^{\frac{-t}{\tau_a} - \frac{B}{\tau_b}e^{\frac{-t}{\tau_b}
\label{eq:generic_from}
\end{equation}

To give the respone defined in Eqn.~\ref{eqn:combined_exponential_response}, coefficients are set $A=-B=f$. The response defined in ***cite*** can then be repeated by setting $\tau_a = \tau_{rise} = 1.7$ and $\tau_b = \tau_{decay} = 0.2$. These coefficients give the response shown in Fig.~\ref{fig:typical_response}.
\begin{figure}
\caption{Synapse behaviour replicating ***cite***, achieved through combining exponetnials according to \ref{eqn:generic_form} and setting coefficients $A=-B=f$ (where $f$ is calculated according to Eqn.~\ref{eqn:combined_exponential_response}), and $\tau_a = 1.7$ and $\tau_b = 0.2$.
\label{fig:typical_response}
\end{figure}
Note that Fig.\ref~{fig:typical_response} also shows the two components before combining: this shows that for the coefficients given, the early response is dominated by the contribution from the first component -- designated $\tau_{rise}$ -- with the subsequent response dominated by the second component -- designated $\tau_{decay}$.  


\section{$\alpha$ Synapyse}
To give the $\alpha$ (alpha) response, coefficients of \ref{eqn:generic_form} are assigned such that $A= , B=$, and hence the overall response collapses to:
\begin{equation}
\label{eqn:alpha_response}
\end{equation}

\end{document}
